\documentclass{resume} % Use the custom resume.cls style

\usepackage[left=0.75in,top=0.6in,right=0.75in,bottom=0.6in]{geometry}
\usepackage{ctex}
\usepackage{fontspec}

\name{李\hspace{0.4cm}岩} % Your name
\address{1990~$\cdot$~09 \\ 浦东新区紫薇路750弄,上海}
\address{(+86)~$\cdot$~185~$\cdot$~2100~$\cdot$~2247 \\ iSeaSoul@gmail.com}

\begin{document}

\begin{rSection}{教育经历}

{\bf 武汉大学} \hfill {2011.9 - 2013.6} \\
计算机软件与理论\ 硕士

{\bf 武汉大学} \hfill {2007.9 - 2011.6} \\
计算机科学与技术\ 本科
\vspace{0.5em}

\end{rSection}

\begin{rSection}{工作实习经验}

\begin{rSubsection}{百度商务搜索部}{2014.11 - 今}{高级研发工程师}{上海,中国}
\item 在“闪投”项目组开发新功能。
\item 百度“凤巢”广告CTR预估模型算法优化。
\item 合作完成了“实时学习”项目,将线上抽取训练样本进入训练流程的时间由3个小时降低到10分钟以内,并使用PU Learning来减小错误样本的影响,模型AUC得到了大幅增长。使用C++/Python。
\end{rSubsection}
\begin{rSubsection}{网易游戏}{2013.7 - 2014.7}{高级开发工程师}{杭州,中国}
\item 在两个月的项目实训中,完成了一个小型但完整的手机卡牌游戏,负责服务端逻辑。
\item 在杭州手游部门参与开发了一款新手游,是该游戏程序开发最初两名程序之一,合作完成了包括游戏引擎、数据库设计、客户端架构与日志系统的大部分工作。使用C++/Python/JavaScript。
\end{rSubsection}
\begin{rSubsection}{微软互联网工程院}{2012.2 - 2012.7}{研发实习生}{北京,中国}
\item 使用ASP.NET MVC 3.0搭建了一个完整的模型训练平台,用来训练Bing搜索中的新闻网页信息抽取模型,训练的新模型的PR值得到了较大提升。
\item 一些小型的数据分析与挖掘工作。使用C\#。
\end{rSubsection}

\end{rSection}

\begin{rSection}{荣誉奖励}
\begin{list}{$\cdot$}{\leftmargin=0em}
\itemsep -0.5em \vspace{0em}
\item Google Code Jam (GCJ) 2014~~现场赛 (全球前25名)  \hfill {2014}
\item 第36届ACM/ICPC华沙全球总决赛~~第18名  \hfill {2012}
\item 第35届ACM/ICPC奥兰多全球总决赛~~第42名  \hfill {2011}
\item 第34 - 36届ACM/ICPC亚洲区域赛~~金奖8次,银奖2次 \hfill {2009 - 2011}
\item 计蒜之道2015程序设计大赛~~现场赛 \hfill {2015}
\item 第二届微软“编程之美”全国挑战赛~~现场赛 \hfill {2013}
\item 腾讯2012编程马拉松大赛~~第二名 \hfill {2012}
\item 华中地区数学建模邀请赛~~一等奖 \hfill {2009}
\item 微软实习生“三国杀”3V3比赛~~第一名 \hfill {2012}  % Just kidding
\item 武汉大学“2011级优秀毕业生”\ 荣誉称号 \hfill {2011}
\end{list}
\vspace{0.5em}

\end{rSection}

\begin{rSection}{技能}

\begin{tabular}{ @{} >{\bfseries}l @{\hspace{6ex}} l }
编程语言 & C/C++, C\#, Python, Java, JavaScript, Scheme, Scala, Shell, \LaTeX \\
工具 & SVN, Git, Vim, Sublime\\
英语 & CET-4, CET-6 \\
其他 & 算法,机器学习,并行计算
\end{tabular}

\end{rSection}

\end{document}
